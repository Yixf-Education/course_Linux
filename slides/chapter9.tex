\input{snippet/beamer_head}
\input{snippet/class_head}

%\includeonlyframes{current}

\title[Perl语言]{第九章\quad Perl语言简介}
\author[Yixf]{伊现富(Yi Xianfu)}
\institute[TIJMU]{天津医科大学(TIJMU)\\ 生物医学工程与技术学院}
\date{2016年4月}


\begin{frame}
  \titlepage
\end{frame}

\begin{frame}[plain,label=current]
  \frametitle{教学提纲}
  \setcounter{tocdepth}{3}
  \begin{multicols}{2}
    \tableofcontents
  \end{multicols}
\end{frame}


\section{引言}
\begin{frame}
  \frametitle{引言 | 简介}
  \begin{block}{Perl}
    \begin{itemize}
      \item Perl是高级、通用、直译式、动态的程序语言
      \item \alert{Practical Extraction and Report Language},实用摘录与报表语言
      \item Pathologically Eclectic Rubbish Lister,病态折中式垃圾列表器
      \item 拉里·沃尔(\alert{Larry Wall}),1987年12月18日
    \end{itemize}
  \end{block}
  \pause
  \begin{block}{特性}
    \begin{itemize}
      \item 具有动态语言的强大灵活的特性
      \item 借用了C、sed、awk、shell等语言的特性,提供了许多冗余语法
      \item 使用了语言学的思维(泛型变量、动态数组、Hash表等)
      \item 程序员可以忽略内部数据存储、类型、内存越界等细节
      \item 内部集成了正则表达式的功能
      \item 巨大的第三方代码库\alert{CPAN}(Comprehensive Perl Archive Network,Perl综合典藏网)
    \end{itemize}
  \end{block}
\end{frame}

\begin{frame}
  \frametitle{引言 | 简介}
  \begin{block}{\alert{书写规范}}
    \begin{itemize}
      \item Perl:程序语言本身
      \item perl:实际编译并运行程序的解释器
      \item PERL:错误的写法,外行的标志
    \end{itemize}
  \end{block}
  \pause
  \begin{block}{应用领域}
    \begin{itemize}
      \item 脚本语言中的瑞士军刀
      \item 网络编程、CGI(Common Gateway Interface,通用网关接口)
      \item 图形编程
      \item 系统管理
      \item 生物信息
      \item \ldots
    \end{itemize}
  \end{block}
\end{frame}

\begin{frame}
  \frametitle{引言 | 标志}
  \begin{figure}
    \centering
    \includegraphics[width=4cm]{c9.perl.logo.01.jpg}
    \hspace{1cm}
    \includegraphics[width=4cm]{c9.perl.logo.02.png}
    \vspace{0.5cm}
    \includegraphics[width=6cm]{c9.perl.logo.03.jpg}
  \end{figure}
\end{frame}

\begin{frame}
  \frametitle{引言 | 中心思想}
  \begin{itemize}
    \item \alert{TMTOWTDI}:There's More Than One Way To Do It. 不只一种方法来做一件事。发音为“Tim Toady”。
    \item TIMTOWTDIBSCINABTE:There's more than one way to do it, but sometimes consistency is not a bad thing either.  不只一种方法来做一件事,但有时保持一致也不错。发音为“Tim Toady Bicarbonate”。
    \item Easy things should be easy, and hard things should be possible. 简单的事情应该是简单的,复杂的事情应该变得可能。
  \end{itemize}
\end{frame}

\begin{frame}
  \frametitle{引言 | 优缺点}
  \begin{block}{优点}
    \begin{itemize}
      \item 很容易:容易使用,但学习Perl并不简单
      \item 几乎不受限制:几乎没有什么事是Perl办不到的
      \item 速度通常很快
    \end{itemize}
  \end{block}
  \pause
  \begin{block}{缺点}
    \begin{itemize}
      \item 灵活、随意和“过度”的冗余语法
      \item write-only:代码有点难看,令人难以阅读
      \item 解释器耗费资源
    \end{itemize}
  \end{block}
\end{frame}

\begin{frame}[fragile]
  \frametitle{引言 | \alert{语法}}
\begin{lstlisting}[language=Perl]
#!/usr/bin/perl
# Classic "Hello World" as done with Perl

use strict;
use warnings;

print "Hello World!\n";
\end{lstlisting}
\pause
\begin{lstlisting}
# Step 1:编写脚本
vim hello.pl
# Step 2:修改权限
chmod 755 hello.pl
# Step 3:运行脚本
./hello.pl
\end{lstlisting}
\end{frame}

\section{变量}
\begin{frame}[fragile]
  \frametitle{变量}
  \begin{block}{变量}
    Perl是一种无类型语言(untyped),换句话说,在语言层面上,Perl和大多数编程语言不同,不把变量分成整数、字符、浮点数等等,而只有一种能接受各种类型数据的“无类型”变量。\\
    Perl中各种变量的运算也很自由,数和含有数的字符串是等效的,可以把数字字符串参与数学计算,也可以反之,让数字参与字符串的构成和操作。
  \end{block}
  \pause
  \begin{block}{\alert{类型}}
    \begin{itemize}
      \item 标量:scalar;只包含一个元素的变量;以\verb|$|开头
      \item 数组:array;含有任意数量元素的变量,以其存储顺序作为索引;以\verb|@|开头
      \item 散列:hash,associative array(关联数组);像字典一样,把不同的变量按照它们的逻辑关系组织起来,并以作为“键”的变量进行索引;以\verb|%|开头
    \end{itemize}
  \end{block}
\end{frame}

\subsection{标量}
\begin{frame}[fragile]
  \frametitle{变量 | 标量 | \alert{语法}}
\begin{lstlisting}[language=Perl]
# 标量:$scalar

# 字符串,双引号
$name = "Paul";

# 数字,无引号
$age = 29;

# 字符串,单引号
$where_to_find_him = 'http://www.weinstein.org';
\end{lstlisting}
\end{frame}

\subsection{数组}
\begin{frame}
  \frametitle{变量 | 数组 | 简介}
  \begin{figure}
    \centering
    \includegraphics[width=10cm]{c9.perl.array.01.jpg}
  \end{figure}
\end{frame}

\begin{frame}[fragile]
  \frametitle{变量 | 数组 | \alert{语法}}
\begin{lstlisting}[language=Perl]
# 数组:@array

# 字符串
@authors = ("Paul", "Joe", "Jeremy", "Harley");
$authors[4] = "Tom";

# 数字
@list = (1, 2, 3, 4);

# 解引用:$array[index]
$authors[4] # Tom
$list[0] # 1
\end{lstlisting}
\end{frame}

\begin{frame}
  \frametitle{变量 | 数组 | \alert{操作}}
  \begin{figure}
    \centering
    \includegraphics[width=10cm]{c9.perl.array.02.jpg}
    \vspace{0.5cm}
    \includegraphics[width=5.5cm]{c9.perl.array.03.png}
    \includegraphics[width=5.5cm]{c9.perl.array.04.png}
  \end{figure}
\end{frame}

\subsection{散列}
\begin{frame}
  \frametitle{变量 | 散列 | 简介}
  \begin{figure}
    \centering
    \includegraphics[width=2.8cm]{c9.perl.hash.01.png}\qquad
    \includegraphics[width=8.2cm]{c9.perl.hash.02.png}
  \end{figure}
\end{frame}

\begin{frame}[fragile]
  \frametitle{变量 | 散列 | \alert{语法}}
\begin{lstlisting}[language=Perl]
# 散列:%hash

# 创建散列
%person = (
name => 'Paul',
age => '29',
url => 'http://www.weinstein.org',
)

# 提取键值:$hash{key}
$person{"age"} # 29
\end{lstlisting}
\end{frame}

\subsection{内置变量}
\begin{frame}
  \frametitle{变量 | 内置变量}
  \begin{figure}
    \centering
    \includegraphics[width=11cm]{c9.perl.buildin.01.jpg}
  \end{figure}
\end{frame}

\section{操作符}
\begin{frame}[fragile]
  \frametitle{操作符 | \alert{数字操作符}}
  \begin{table}
    \centering
    \rowcolors[]{1}{blue!20}{blue!10}
    \begin{tabularx}{0.7\textwidth}{cX}
      \hline
      \rowcolor{blue!50}操作符 & 含义\\
      \hline
      \verb|+| & 加法\\
      \verb|-| & 减法\\
      \verb|*| & 乘法\\
      \verb|/| & 除法\\
      \verb|**| & 乘幂,乘方\\
      \% & 取模,取余\\
      \hline
      \verb|<|  & 小于\\
      \verb|>|  & 大于\\
      \verb|==|  & 等于\\
      \verb|<=|  & 小于等于\\
      \verb|>= |  & 大于等于\\
      \verb|!=|  & 不等于\\
      \verb|<=>|  & 比较。\verb|a<=>b|:a等于b时返回0,a大于b时返回1,a小于b时返回-1\\
      \hline
    \end{tabularx}
  \end{table}
\end{frame}

\begin{frame}
  \frametitle{操作符 | \alert{字符串操作符}}
  \begin{table}
    \centering
    \rowcolors[]{1}{blue!20}{blue!10}
    \begin{tabularx}{0.7\textwidth}{cX}
      \hline
      \rowcolor{blue!50}操作符 & 含义\\
      \hline
      \verb|.| & 连接。\verb|"string1" . "string2"|\\
      \verb|x| & 重复。\verb|"string" x number|\\
      \verb|lt| & 小于\\
      \verb|gt| & 大于\\
      \verb|eq| & 等于\\
      \verb|le| & 小于等于\\
      \verb|ge| & 大于等于\\
      \verb|ne| & 不等于\\
      \verb|cmp| & 比较。类似于数字比较的\verb|<=>|\\
      \hline
    \end{tabularx}
  \end{table}
\end{frame}

\begin{frame}
  \frametitle{操作符 | \alert{逻辑操作符}}
  \begin{table}
    \centering
    \rowcolors[]{1}{blue!20}{blue!10}
    \begin{tabularx}{0.5\textwidth}{cX}
      \hline
      \rowcolor{blue!50}操作符 & 含义\\
      \hline
      \verb|&&| & 逻辑AND,与\\
      \verb=||= & 逻辑OR,或\\
      \verb|!| & 逻辑NOT,非\\
      \verb|?=| & 条件操作符\\
      \hline
    \end{tabularx}
  \end{table}
\end{frame}

\begin{frame}
  \frametitle{操作符 | 文件测试操作符}
  \begin{table}
    \centering
    \rowcolors[]{1}{blue!20}{blue!10}
    \begin{tabularx}{0.7\textwidth}{cX}
      \hline
      \rowcolor{blue!50}操作符 & 含义\\
      \hline
      -r & 可读\\
      -w & 可写\\
      -x & 可执行\\
      -e & 存在\\
      -z & 存在但没有内容\\
      -s & 存在且有内容\\
      -f & 普通文件\\
      -d & 目录\\
      -l & 符号链接\\
      -T & 看起来像文本文件\\
      -B & 看起来像二进制文件\\
      -M & 最后被修改后至今的天数\\
      -A & 最后被访问后至今的天数\\
      -C & 最后inode变更后至今的天数\\
      \hline
    \end{tabularx}
  \end{table}
\end{frame}

\begin{frame}[fragile]
  \frametitle{操作符 | \alert{匹配操作符}}
  \begin{table}
    \centering
    \rowcolors[]{1}{blue!20}{blue!10}
    \begin{tabularx}{0.5\textwidth}{cX}
      \hline
      \rowcolor{blue!50}操作符 & 含义\\
      \hline
      \verb|=~| & 绑定操作符,匹配\\
      \verb|!~| & 绑定操作符,不匹配\\
      \verb|~~| & 智能匹配操作符\\
      \hline
    \end{tabularx}
  \end{table}
\end{frame}

\section{基本函数}
\begin{frame}[fragile]
  \frametitle{基本函数 | \alert{print,chomp}}
\begin{lstlisting}[language=Perl]
# 向标准输出打印文本
print "Hello Again\n";

# 向一个具有文件句柄的文件打印文本
print FILE "Hello Again\n";

# 打印变量的值
print "How are on this day, the " . $date . "?\n";
\end{lstlisting}
\pause
\begin{lstlisting}[language=Perl]
# 删除变量末尾的(多个)换行符,返回删除的换行符的个数
chomp $name;
chomp @authors;
\end{lstlisting}
\end{frame}

\begin{frame}[fragile]
  \frametitle{基本函数 | \alert{join,split}}
\begin{lstlisting}[language=Perl]
# joining a number of strings togother with a colon delimiter
$fields = join ':', $data_field1, $data_field2, $data_field3;

# splitting a string into substrings
($field1, $field2) = split /:/, 'Hello:World', 2;

# splitting a scalar and creating an array
@fields = split /:/, $raw_data;
\end{lstlisting}
\end{frame}

\begin{frame}[fragile]
  \frametitle{基本函数 | \alert{open,close}}
\begin{lstlisting}[language=Perl]
# open the file and slurp its contents into an array and then close the file
open(FILE, "/etc/passwd");
@filedata = <FILE>;
close(FILE);

open my $IN, '<', $file_in or die "$0 : failed to open input file '$file_in' : $!\n";
while(<$IN>){
  chomp;
  actions;
}
close $IN or warn "$0 : failed to close input file '$file_in' : $!\n";
\end{lstlisting}
\end{frame}

\begin{frame}[fragile]
  \frametitle{基本函数 | opendir,readdir,closedir}
\begin{lstlisting}[language=Perl]
#!/usr/bin/perl -w

print "Read user's home directory\n";
opendir(HOMEDIR, ".");
@ls = readdir HOMEDIR;
closedir(HOMEDIR);

print "Create file dirlist.txt with a directory listing of user's home dir\n";
open(FILE, ">dirlist.txt");
foreach $item (@ls) {
  print FILE $item . "\n";
}
close(FILE);
print "All done\n\n";
\end{lstlisting}
\end{frame}

\begin{frame}[fragile]
  \frametitle{基本函数 | \alert{my},local}
\begin{lstlisting}[language=Perl]
# Global variable $name is given a name
$name = "Paul";

# Enter our loop
foreach (@filedata) {
  # declare a new variable for just the loop
  my $current_file;

  # create a local version of name to temporarily assign values within the loop to
  local $name;

  ...
}
\end{lstlisting}
\end{frame}

\section{判断语句}
\subsection{if语句}
\begin{frame}
  \frametitle{判断语句 | if | 逻辑流程}
  \begin{figure}
    \centering
    \includegraphics[width=5cm]{c9.perl.if.01.jpg}
  \end{figure}
\end{frame}

\begin{frame}[fragile]
  \frametitle{判断语句 | if | \alert{语法}}
\begin{lstlisting}[language=Perl]
# if区块
if ($hour > 22) {
  print "should sleep...\n";
}

# if语句
print "hello" if $guest >= 1;
\end{lstlisting}
\end{frame}

\begin{frame}
  \frametitle{判断语句 | if-else | 逻辑流程}
  \begin{figure}
    \centering
    \includegraphics[width=6cm]{c9.perl.if.else.01.png}
    \includegraphics[width=5cm]{c9.perl.elsif.01.png}
  \end{figure}
\end{frame}

\begin{frame}[fragile]
  \frametitle{判断语句 | if-else | \alert{语法}}
\begin{lstlisting}[language=Perl]
if ($name eq "Paul") {
  print "Hi Paul\n";
}
elsif ($name eq "Joe") {
  print "Hi Joe\n";
}
elsif ($name eq "Jeremy") {
  print "Hi Jeremy\n";
}
else {
  print "Sorry, have we meet before?";
}
\end{lstlisting}
\end{frame}

\subsection{unless语句}
\begin{frame}
  \frametitle{判断语句 | unless | 逻辑流程}
  \begin{figure}
    \centering
    \includegraphics[width=3.5cm]{c9.perl.unless.01.png}
  \end{figure}
\end{frame}

\begin{frame}[fragile]
  \frametitle{判断语句 | unless | \alert{语法}}
\begin{lstlisting}[language=Perl]
# unless区块
unless ($credit > 100) {
  print "You can not graduate!\n";
}

# unless语句
print "eat\n" unless $food==0;
\end{lstlisting}
\end{frame}

\subsection{given-when语句}
\begin{frame}
  \frametitle{判断语句 | given-when | 逻辑流程}
  \begin{figure}
    \centering
    \includegraphics[width=4.5cm]{c9.perl.given.01.png}
  \end{figure}
\end{frame}

\begin{frame}[fragile]
  \frametitle{判断语句 | given-when | 语法}
\begin{lstlisting}[language=Perl]
use 5.010;
given ($foo) {
  say "a" when "a";
  when (/b/) {say "b";}
  default {say "not match";}
}
\end{lstlisting}
\end{frame}

\section{循环语句}
\subsection{foreach语句}
\begin{frame}
  \frametitle{循环语句 | foreach | 逻辑流程}
  \begin{figure}
    \centering
    \includegraphics[width=6cm]{c9.perl.foreach.01.jpg}
  \end{figure}
\end{frame}

\begin{frame}[fragile]
  \frametitle{循环语句 | foreach | \alert{语法}}
\begin{lstlisting}[language=Perl]
@group = 1..10;

# foreach循环
foreach my $element (@group) {
  print "$element\n";
}

# 等价的for循环
for (@group) {
  print "$_\n";
}
print "$_\n" for @group;
\end{lstlisting}
\end{frame}

\subsection{for语句}
\begin{frame}
  \frametitle{循环语句 | for | 逻辑流程}
  \begin{figure}
    \centering
    \includegraphics[width=5cm]{c9.perl.for.01.png}
  \end{figure}
\end{frame}

\begin{frame}[fragile]
  \frametitle{循环语句 | for | \alert{语法}}
\begin{lstlisting}[language=Perl]
# 从1数到10
for ($i = 1; $i <= 10; $i++) {
  print "I can count to $i!\n";
}
\end{lstlisting}
\end{frame}

\subsection{while语句}
\begin{frame}
  \frametitle{循环语句 | while | 逻辑流程}
  \begin{figure}
    \centering
    \includegraphics[width=5cm]{c9.perl.while.01.png}
  \end{figure}
\end{frame}

\begin{frame}[fragile]
  \frametitle{循环语句 | while | \alert{语法}}
\begin{lstlisting}[language=Perl]
$i = 0;
while ($i < 10) {
  print "$i\n";
  $i++;
}
\end{lstlisting}
\end{frame}

\begin{frame}
  \frametitle{循环语句 | do-while | 逻辑流程}
  \begin{figure}
    \centering
    \includegraphics[width=5cm]{c9.perl.do.while.01.jpg}
  \end{figure}
\end{frame}

\begin{frame}[fragile]
  \frametitle{循环语句 | do-while | 语法}
\begin{lstlisting}[language=Perl]
$i = 0;
do {
  print "$i\n";
  $i = $i + 1;
} while ($i < 10);
\end{lstlisting}
\end{frame}

\subsection{until语句}
\begin{frame}
  \frametitle{循环语句 | until | 逻辑流程}
  \begin{figure}
    \centering
    \includegraphics[width=5cm]{c9.perl.until.01.png}
  \end{figure}
\end{frame}

\begin{frame}[fragile]
  \frametitle{循环语句 | until | \alert{语法}}
\begin{lstlisting}[language=Perl]
$i = 0;
until ($i == 10) {
  print "$i\n";
  $i++;
}
\end{lstlisting}
\end{frame}

\begin{frame}
  \frametitle{循环语句 | do-until | 逻辑流程}
  \begin{figure}
    \centering
    \includegraphics[width=4cm]{c9.perl.do.until.01.png}
  \end{figure}
\end{frame}

\begin{frame}[fragile]
  \frametitle{循环语句 | do-until | 语法}
\begin{lstlisting}[language=Perl]
$i = 0;
do {
  print "$i\n";
  $i++;
} until ($i == 10);
\end{lstlisting}
\end{frame}

\section{检修脚本}
\begin{frame}[fragile]
  \frametitle{\alert{检修脚本}}
\begin{lstlisting}[language=Perl]
# 不洁模式
#!/usr/bin/perl -T

# 打开警告
use warnings;

# 严格模式,语法更加规范
use strcit;
\end{lstlisting}
\end{frame}

\begin{frame}[fragile]
  \frametitle{\alert{检修脚本}}
\begin{lstlisting}
# 检查语法
perl -c script.pl

# 格式化脚本
perltidy script.pl

# 调试脚本
perl -d script.pl
\end{lstlisting}
\end{frame}

\begin{frame}
  \frametitle{检修脚本 | 调试}
  \begin{table}
    \centering
    \rowcolors[]{1}{blue!20}{blue!10}
    \begin{tabularx}{0.8\textwidth}{cX}
      \hline
      \rowcolor{blue!50}命令 & 功能\\
      \hline
      s & 仅执行脚本中的一行\\
      n & 按步执行以避免进入子程序\\
      c & 继续运行直至遇到断点\\
      r & 继续运行直至从当前子程序中执行返回命令\\
      w & 显示当前行前后的代码\\
      b & 设置断点\\
      x & 显示变量的值\\
      W & 设置监视表达式\\
      T & 显示栈回溯追踪\\
      L & 列出所有的断点\\
      D & 删除所有的断点\\
      R & 重新启动脚本以便再次测试\\
      \hline
    \end{tabularx}
  \end{table}
\end{frame}


\section{回顾与总结}
\subsection{总结}
\begin{frame}
  \frametitle{Perl语言 | 总结}
  \begin{block}{知识点}
    \begin{itemize}
      \item Perl语言简介:中心思想,优缺点,语法结构
      \item 变量:标量,数组,散列,内置变量
      \item 操作符:数字、字符串、逻辑、文件测试、匹配操作符
      \item 基本函数:print, chomp, join, split, open, close, my
      \item 判断语句:if, unless, given-when
      \item 循环语句:foreach, for, while, until
      \item 检修脚本:检查语法,格式化脚本,调试脚本
    \end{itemize}
  \end{block}
  \begin{block}{技能}
    \begin{itemize}
      \item 掌握Perl语言的基本语法
      \item 使用Perl编写简单的应用脚本
    \end{itemize}
  \end{block}
\end{frame}

\subsection{思考题}
\begin{frame}
  \frametitle{Perl语言 | 思考题}
  \begin{enumerate}
    \item Perl语言的中心思想是什么?
    \item Perl中的变量主要有哪三大类?
    \item 列举Perl中的各种操作符。
    \item chomp, join, split的作用是什么?
    \item 在Perl中如何和文件进行交互?
    \item Perl中的判断语句有哪些,语法是怎样的?
    \item Perl中的循环语句有哪些,语法是怎样的?
    \item 如何调试Perl脚本?
  \end{enumerate}
\end{frame}


\input{snippet/class_tail}
