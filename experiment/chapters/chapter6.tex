\chapter{Linux中的软件管理}

%\vspace{0.2in}
\noindent
一、实验要求
\begin{enumerate}
  \item 掌握在命令行中下载文件的方法。
  \item 掌握使用APT与Yum管理软件的方法。
  \item 掌握使用dpkg与RPM管理软件的方法。
  \item 掌握通过源代码安装软件的步骤。
  \item 熟悉其他安装软件的方法。
\end{enumerate}

\vspace{0.2in}
\noindent
二、实验准备
\begin{enumerate}
  \item 安装有CentOS 7的计算机。
  \item 安装有Ubuntu 14.10的计算机。
\end{enumerate}

\vspace{0.2in}
\noindent
三、实验内容

\vspace{0.1in}
(一)在命令行中下载安装包。

以datamash为例,使用wget和curl在命令行中下载其安装包。
\begin{lstlisting}[language=bash]
# wget
wget -c http://files.housegordon.org/datamash/bin/datamash_1.0.6-1_amd64.deb
wget -c http://ftp.gnu.org/gnu/datamash/datamash-1.0.6.tar.gz
# curl
## 使用原始文件名
curl -O http://files.housegordon.org/datamash/bin/datamash-1.0.6-1.el6.x86_64.rpm
## 使用新的文件名
curl -o datamash.tar.gz http://ftp.gnu.org/gnu/datamash/datamash-1.0.6.tar.gz
\end{lstlisting}

\vspace{0.1in}
(二)通过APT与Yum安装软件。

以htop(或Glances,dos2unix等)为例,使用APT在Ubuntu中安装htop,使用Yum在CentOS中安装htop。
\begin{lstlisting}[language=bash]
# 需要使用sudo
apt-get install htop
apt-get install glances
apt-get install dos2unix

# 需要切换为root用户
yum install htop
yum install glances
yum install dos2unix
\end{lstlisting}

\vspace{0.1in}
(三)通过dpkg与RPM安装软件。

以Webmin(或datamash,TeamViewer,RStudio等)为例,使用dpkg在Ubuntu中安装Webmin,使用RPM在CentOS中安装Webmin。
\begin{lstlisting}[language=bash]
# 需要使用sudo
dpkg -i webmin_1.700_all.deb
dpkg -i datamash_1.0.6-1_amd64.deb
dpkg -i teamviewer_linux.deb
dpkg -i rstudio-0.98.1062-amd64.deb

# 需要切换为root用户
rpm -ivh webmin-1.700-1.noarch.rpm
rpm -ivh datamash-1.0.6-1.el6.x86_64.rpm
rpm -ivh teamviewer_linux.rpm
rpm -ivh rstudio-0.98.1062-x86_64.rpm
\end{lstlisting}

\vspace{0.1in}
(四)通过源代码安装软件。

以dos2unix(或datamash,htop,parallel,FASTX-Toolkit,msort,SAMtools,BEDTools,seqtk等)为例,使用源代码对其进行安装。
\begin{lstlisting}[language=bash]
# dos2unix
tar -zxvf dos2unix-7.0.tar.gz
cd dos2unix-7.0/
vim INSTALL.txt
make
make check
#make strip
make install
#make clean
#make mostlyclean

# datamash
tar -zxvf datamash-1.0.6.tar.gz
cd datamash-1.0.6/
vim INSTALL
vim README
./configure
make
make check
sudo make install
#make installcheck
#make clean
#make uninstall

# htop
tar -zxvf htop-1.0.3.tar.gz
cd htop-1.0.3/
vim INSTALL
vim README
./configure
make
make install

# parallel
tar -xjvf parallel-20140822.tar.bz2
cd parallel-20140822/
vim README
./configure
make
make install

# FASTX-Toolkit
tar -xjvf fastx_toolkit-0.0.14.tar.bz2
cd fastx_toolkit-0.0.14/
vim INSTALL
vim README
./configure
make
sudo make install

# msort
tar -zxvf msort.tar.gz
cd msort/
vim README
./autogen.sh                                                              
./configure
make

# SAMtools
tar -xjvf samtools-1.0.tar.bz2
cd samtools-1.0/
vim INSTALL
make
make install

# BEDTools
tar -zxvf bedtools-2.21.0.tar.gz
cd bedtools2/
make
sudo cp ./bin/* /usr/local/bin

# seqtk
unzip seqtk-master.zip
cd seqtk-master/
make
\end{lstlisting}

\vspace{0.1in}
(五)通过脚本安装软件。

以cheat(或Glances,Webmin等)为例,使用脚本对其进行安装。
\begin{lstlisting}[language=bash]
# cheat
## First install the required python dependencies
sudo pip install docopt pygments
## Then
unzip cheat-master.zip
cd cheat-master
vim README.md
sudo python setup.py install

# Glances
unzip Glances-master.zip
cd Glances-master
vim README.rst
python setup.py install

# Webmin
tar -zxvf webmin-1.700.tar.gz
cd webmin-1.700/
vim README
./setup.sh
\end{lstlisting}

\vspace{0.1in}
(六)通过其他方式安装软件。

以Galaxy(或cheat,Glances等)为例,根据安装说明对其进行安装。
\begin{lstlisting}[language=bash]
# Galaxy
cd ~
hg clone https://bitbucket.org/galaxy/galaxy-dist/
cd galaxy-dist
hg update stable

# cheat
## Using pip
sudo pip install cheat
## Using homebrew
brew install cheat

# Glances
pip install Glances
\end{lstlisting}

\vspace{0.1in}
(七)不需要安装的软件。

以CPU-G(或FASTX-Toolkit,WebLogo,TeamViewer,IGV等)为例,下载后可以直接使用。
\begin{lstlisting}[language=bash]
# CPU-G
tar -zxvf cpu-g-0.9.0.tar.gz
cd cpu-g-0.9.0/
chmod 755 cpu-g
./cpu-g

# FASTX-Toolkit
tar -xjvf fastx_toolkit_0.0.13_binaries_Linux_2.6_amd64.tar.bz2
cd bin

# WebLogo
tar -zxvf weblogo-3.3.tar.gz
cd weblogo-3.3/
./weblogo -h

# TeamViewer
tar -zxvf teamviewer_linux.tar.gz
cd teamviewer9/
./teamviewer

# IGV
unzip IGV_2.3.34.zip
cd IGV_2.3.34/
vim readme.txt
./igv.sh
\end{lstlisting}

\vspace{0.1in}
(八)从源代码编译openssl程序。
\begin{enumerate}
  \item 在主目录中,创建一个目录(例如src),然后切换到这个目录,软件的编译将在这个目录中进行。
  \item 获取openssl的源代码。
  \item 提取源代码并切换到新目录中。
  \item 使用ls命令查看包含在安装目录中的文件(有多个README*和INSTALL*文件)。
  \item 使用less或more命令阅读这些找到的文件。
  \item 根据INSTALL文件中的说明,运行带有\verb|--prefix|开关的config脚本。
  \item 开始编译:\verb|make|。
  \item 为了在安装之前测试已编译好的程序,运行带有test的make命令:\verb|make test|。
  \item 安装软件:\verb|make install|。
\end{enumerate}

\vspace{0.1in}
(九)编译需要预装软件的代码。
\begin{enumerate}
  \item 切换到源代码编译目录。
  \item 找到Lynx的源代码。
  \item 提取源代码并切换到新创建的目录中。
  \item 查找README或INSTALL文件。
  \item 运行\verb|./configure --help|查看编译选项列表。
  \item 必须执行一个特殊的步骤,这个步骤配置Lynx编译程序以便安装到主目录中,并使用前面已经安装的SSL库。
  \item 准备安装:\verb|make|。
  \item 为了安装软件,执行命令:\verb|make install|。
  \item 如果希望安装其他Lynx文档,可以根据屏幕上给出的提示进行操作。
\end{enumerate}

\vspace{0.1in}
(十)使用RPM。
\begin{enumerate}
  \item 在命令行中输入以下命令:\verb=rpm -qa | grep mysql=。根据个人的安装情况,要么会获得一些输出,显示在机器上安装了哪个版本的MySQL,要么什么输出都没有。
  \item 要安装新版的MySQL,需要从MySQL的网站下载相关的RPM包并保存到所需要的文件夹中。
  \item 执行命令:\verb|rpm -Uvh MySQL.rpm|。
\end{enumerate}

\vspace{0.1in}
(十一)软件管理。

尝试使用dpkg与APT、RPM与Yum对软件(如:htop,Glances,dos2unix等)进行管理(如:查询、安装、卸载等)。

