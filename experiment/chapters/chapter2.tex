\chapter{Linux图形界面的基本操作}

{\itshape
目前,Linux操作系统上最常用的桌面环境主要是Unity、GNOME和KDE,此外,还有Xfce、LXDE等多种桌面环境。Ubuntu 14.10以Unity作为默认的桌面环境,CentOS 7以GNOME作为默认的桌面环境。本实验以CentOS 7为基础,来练习GNOME桌面环境的基本操作。
}

\vspace{0.2in}
\noindent
一、实验要求
\begin{enumerate}
  \item 了解GNOME系统面板的设置方法。
  \item 掌握GNOME桌面环境的设置方法。
  \item 掌握GNOME启动项的新建方法。
  \item 掌握输入法的设置方法。
  \item 掌握文件浏览器的使用方法。
  \item 掌握桌面环境下管理用户与组的方法。
  \item 了解其他的桌面环境。
\end{enumerate}

\vspace{0.2in}
\noindent
二、实验准备
\begin{enumerate}
  \item 安装有CentOS 7的计算机。
\end{enumerate}

\vspace{0.2in}
\noindent
三、实验内容

\vspace{0.1in}
(一)设置面板
\begin{enumerate}
  \item 设置底部面板可隐藏。
    \begin{enumerate}
      \item 以普通用户身份登录CentOS 7,进入GNOME桌面环境。
      \item 右击底部面板的空白处,弹出快捷菜单,选择“属性”命令,弹出“面板属性”对话框。
      \item 在“常规”选项卡中选中“自动隐藏”复选框,并选中“显示隐藏按钮”复选框。最后单击“关闭”按钮,底部面板将自动隐藏。
      \item 移动光标到桌面的下边缘,出现底部面板。此时,底部面板的左右两端出线两个隐藏的按钮。单击左隐藏按钮,面板向左侧收缩。再单击左隐藏按钮,系统面板复原。
      \item 再次设置底部面板,恢复其默认状态。
    \end{enumerate}
  \item 在顶部面板上添加、移动和删除对象。
    \begin{enumerate}
      \item 右击顶部面板的空白处,弹出快捷菜单,选择“添加到面板”命令,弹出“添加到面板”对话框。
      \item 选择“系统监视器”选项,单击“添加”按钮,则顶部面板出现黑底的系统监视器图标。单击这一图标,将打开“系统监视器”窗口。
      \item 继续从“添加到面板”对话框中选择“抽屉”选项,单击“添加”按钮,可在面板上添加一个抽屉。单击“添加到面板”对话框中的“关闭”按钮,关闭“添加到面板”对话框。
      \item 单击抽屉图标,可打开抽屉。
      \item 右击顶部面板的系统监视器图标,弹出快捷菜单,选择“移动”命令,此时系统监视器图标便会跟随鼠标移动。将其移动到抽屉中,单击鼠标左键,系统监视器图标将固定到抽屉中。单击向上箭头,可关闭抽屉。
      \item 右击抽屉图标,弹出快捷菜单,选择“从面板上删除”命令,出现提示信息。需要注意的是:删除抽屉将删除抽屉中的所有内容。单击“删除”按钮,连同系统监视器图标一起删除。
    \end{enumerate}
\end{enumerate}

\vspace{0.1in}
(二)设置桌面
\begin{enumerate}
  \item 将桌面背景设置为瓢虫图片。
    \begin{enumerate}
      \item 右击桌面空白处,弹出快捷菜单,选择“更改桌面背景”命令,弹出“外观首选项”对话框,选择“背景”选项卡。
      \item 选择有小瓢虫的图片,则所有工作区的桌面背景都发生变化。最后单击“关闭”按钮,关闭“外观选项卡”对话框。
    \end{enumerate}
  \item 设置屏幕保护程序。
    \begin{enumerate}
      \item 依次选择“系统”$\rightarrow$“首选项”$\rightarrow$“屏幕保护程序”命令,弹出“屏幕保护程序首选项”对话框。
      \item 选择“宇宙”主题,右侧预览框将显示屏幕保护时随机出现的与宇宙相关的图片。
      \item 拖动时间滑块将“于此事件后视计算机为空闲”设置为1分钟,并保持“计算机空闲时激活屏幕保护程序”复选框和“屏幕保护程序激活时锁定屏幕”复选框处于选中状态,即设置为1分钟后锁定屏幕,最后单击“关闭”按钮。
      \item 静置计算机,1分钟后可观察到屏幕保护程序的效果,此时对计算机进行任何操作都将弹出对话框,要求输入用户的密码,密码验证成功才能回到桌面环境。
    \end{enumerate}
\end{enumerate}

\vspace{0.1in}
(三)设置桌面图标
\begin{enumerate}
  \item 新建“我的文档”文件夹图标。
    \begin{enumerate}
      \item 右击桌面空白处,弹出快捷菜单,选择“创建文件夹”命令,桌面出现一个新的文件夹,其名称默认为“未命名文件夹”。
      \item 按【Ctrl+Space】组合键,启动中文输入法,将文件夹名修改为“我的文档”,并按【Enter】键确认。
    \end{enumerate}
  \item 新建文本编辑器gedit的启动器(图标)。
    \begin{enumerate}
      \item 右击桌面空白处,弹出快捷菜单,选择“创建启动器”命令,弹出“创建启动器”对话框。
      \item 在“名称”文本框中输入应用程序快捷图标的名字“gedit”,在“命令”文本框中输入文本编辑器程序的路径“/usr/bin/gedit”。
      \item 单击图标按钮,弹出“选择图标”对话框,选中/usr/share/pixmaps目录中的apple-red.png文件作为图标。单击“打开”按钮,回到“创建启动器”对话框。
      \item 单击“确定”按钮,桌面多出一个应用程序快捷图标,双击即可打开gedit文本编辑器。
    \end{enumerate}
\end{enumerate}

\vspace{0.1in}
(四)设置主题
\begin{enumerate}
  \item 创建新主题。
    \begin{enumerate}
      \item 依次选择“系统”$\rightarrow$“首选项”$\rightarrow$“外观”命令,弹出“外观首选项”对话框,CentOS 7默认采用System主题。
      \item 单击“自定义”按钮,出现所有可用的主题细节信息,从“窗口边框”选项卡中选中Crux选项,注意边框样式的变化。
      \item 从“图标”选项卡中选择“十字架”选项,注意桌面上文件夹图标的变化。单击“关闭”按钮,回到“主题”选项卡。
      \item 此时所有主题列表的最前面出现一个新的“自定义主题”。单击“另存为”按钮,弹出“主题另存为”对话框,在“名称”文本框中输入主题名称,如“我的主题”,在“描述”文本框中输入对此主题的描述信息。最后单击“保存”按钮,回到“主题”选项卡。
      \item 此时在“主题”选项卡中可查看到刚定义的主题“我的主题”,并按照字母顺序排列在主题列表中。
    \end{enumerate}
\end{enumerate}

\vspace{0.1in}
(五)增加启动项
\begin{enumerate}
  \item 实现登录桌面环境就自动启动文本编辑器(/usr/bin/gedit)。
    \begin{enumerate}
      \item 依次选择“系统”$\rightarrow$“首选项”$\rightarrow$“启动应用程序”命令,弹出“启动应用程序首选项”对话框。
      \item 单击“添加”按钮,弹出“添加启动程序”对话框。在“名称”文本框中输入启动程序名,如“文本编辑器”,在“命令”文本框中输入文本编辑器的路径“/usr/bin/gedit”。单击“添加”按钮,回到“启动应用程序首选项”对话框,此时文本编辑器命令行将出现在“额外的启动程序”列表中。最后单击“关闭”按钮。
      \item 选择“系统”菜单中的“注销”命令,并单击“注销”按钮,当前用户退出。重新登录,可检查系统是否自动启动文本编辑器。
    \end{enumerate}
\end{enumerate}

\vspace{0.1in}
(六)设置输入法

GNOME桌面环境在文本编辑器的编辑区域内,默认使用键盘输入的是英文字母和字符。按【Ctrl+Space】组合键,将切换到中文输入法。
\begin{enumerate}
  \item 仅保留拼音输入法。
    \begin{enumerate}
      \item 右击桌面空白处,弹出快捷菜单,从中选择“创建文档”$\rightarrow$“空文件”命令创建一个空白文档,并将此文件命名为f1。
      \item 双击f1文件,系统自动打开gedit文本编辑器。按【Ctrl+Space】组合键,切换至中文输入法。此时顶部面板出现“拼”字图标,表示当前采用“汉语-Pinyin”输入法。单击“拼”字图标显示所有可用的输入法。在gedit文本编辑器中输入任何英文字符,将显示出中文输入条。
      \item 对于大多数用户而言,只需要保留一种自己操作最熟练的输入法即可。依次选择“系统”$\rightarrow$“首选项”$\rightarrow$“输入法”命令,弹出“IM Chooser-输入法配置工具”对话框。单击“首选输入法”按钮,弹出“IBUS设置”对话框。
      \item 选择“输入法”选项卡。选中不需要的输入法,单击“删除”按钮,最后单击“关闭”按钮。此时,单击顶部面板的“拼”字图标,显示只有“汉语-Pinyin”输入法可用。
    \end{enumerate}
  \item 设置拼音输入法,启动模糊音。
    \begin{enumerate}
      \item 单击顶部面板的“拼”字图标,从展开的菜单中选择“拼音首选项”命令,显示拼音输入法的初始状态和外观特征。
      \item 单击“模糊音”选项卡,选中“启动模糊音”复选框,可进一步选择允许哪些模糊音。
      \item 设置完成后,单击“关闭”按钮,并移动f1文件至用户主文件夹,为后续操作做准备。
    \end{enumerate}
\end{enumerate}

\vspace{0.1in}
(七)使用文件浏览器
\begin{enumerate}
  \item 基本文件操作。
    \begin{enumerate}
      \item 双击桌面上的用户主文件夹图标,启动文件浏览器,观察窗口的各组成部分,并可发现新建的f1文件。
      \item 右击f1文件,弹出快捷菜单,选择“复制”命令,然后在窗口空白处再次右击,从弹出的快捷菜单中选择“粘贴”命令,文件浏览器中多出一个文件,名为“f1(复件)”。
      \item 右击f1文件,弹出快捷菜单,选择“创建链接”命令,窗口中多出一个链接文件,名为“到f1的链接”。
      \item 右击“f1(复件)”文件,弹出快捷菜单,选择“重命名”命令,文件名变为可编辑的文本框,输入新的文件名f2,然后单击窗口空白处。
      \item 在窗口的空白处右击,弹出快捷菜单,选择“创建文件夹”命令,出现一个文件夹,默认名为“未命名文件夹”,文件名处于可编辑状态,输入新文件夹名backup。
      \item 拖动f2文件至backup文件夹中实现文件的移动。
      \item 右击f1文件,弹出快捷菜单,选择“属性”命令。弹出“f1属性”对话框,选中“徽标”选项卡中的urgent徽标。此时。f1文件图标上出现紧急徽标。
    \end{enumerate}
  \item 查看隐藏文件。
    \begin{enumerate}
      \item 选择“查看”菜单中的“显示隐藏文件”命令。
      \item 文件浏览器窗口中多出一些目录和文件。这些文件的文件名都以“.”开头,是Linux中的隐藏文件。
    \end{enumerate}
\end{enumerate}

\vspace{0.1in}
(八)桌面环境下管理用户与组。
\begin{enumerate}
  \item 新建两个用户账号,其用户名为xuser1和xuser2,密码为e12ut59er和wfu1t28er。
    \begin{enumerate}
      \item 以超级用户身份登录X Window图形化用户界面,依次选择“系统”$\rightarrow$“管理”$\rightarrow$“用户和组群”命令,打开“用户管理者”窗口。
      \item 单击工具栏中的“添加用户”按钮,打开“添加新用户”窗口。在“用户名”文本框中输入用户名xuser1,在“密码”文本框中输入密码e12ut59er,在“确认密码”文本框中再次输入密码,然后单击“确定”按钮,返回“用户管理者”窗口。
      \item 用同样的方法新建用户xuser2。
      \item 依次选择“应用程序”$\rightarrow$“附件”$\rightarrow$“gedit文本编辑器”命令,启动文本编辑器,打开/etc/passwd和/etc/shadow文件,将发现文件末尾均出现表示xuser1和xuser2用户账号的信息。打开/etc/group和/etc/gshadow文件,将发现文件末尾均出现表示xuser1和xuser2私人组的信息。
      \item 按【Ctrl+Alt+F2】组合键切换到第二个虚拟终端,输入用户名xuser2和相应的密码可登录系统,说明新建用户操作已成功。
      \item 输入命令pwd,屏幕显示用户登录系统后,自动进入用户主目录“/home/xuser2”。
      \item 输入命令exit,xuser2用户退出登录。
      \item 按【Ctrl+Alt+F1】组合键返回GNOME桌面环境。
    \end{enumerate}
  \item 锁定xuser2用户账号。
    \begin{enumerate}
      \item 在“用户管理者”窗口选中xuser2用户账户,单击工具栏中的“属性”按钮,打开“用户属性”窗口。
      \item 选中“账户信息”选项卡,选中“本地密码被锁”复选框。单击“确定”按钮,返回“用户管理者”窗口。
      \item 按【Ctrl+Alt+F2】组合键,再次切换到第二个虚拟终端,输入用户名xuser2和相应的密码,发现xuser2用户无法登录系统,说明xuser2用户账号已被锁定。
      \item 再次返回GNOME桌面环境。
    \end{enumerate}
  \item 删除xuser2用户。
    \begin{enumerate}
      \item 在“用户管理者”窗口中,选择“编辑”$\rightarrow$“首选项”命令,弹出“首选项”对话框,不选中“隐藏系统用户和组”复选框,最后单击“关闭”按钮。此时“用户”选项卡中显示包括超级用户和系统用户在内的所有用户的账号信息。
      \item 在“搜索过滤器”文本框中输入“x*”并按【Enter】键,则仅显示用户名以x为首字母的用户。
      \item 选中xuser2用户,单击工具栏中的“删除”按钮,单击“是”按钮,返回“用户管理者”窗口,发现xuser2用户已被删除。
      \item 在“搜索过滤器”文本框中输入“*”并按【Enter】键,则显示所有用户。
    \end{enumerate}
  \item 新建两个组,分别是myusers和temp。
    \begin{enumerate}
      \item 在“用户管理者”窗口选中“组群”选项卡,显示出所有组。
      \item 单击工具栏中的“添加组群”按钮,打开“添加新组群”窗口。在“组群名”文本框中输入musers,单击“确定”按钮,返回“用户管理者”窗口。
      \item 用相同的方法新建temp组。
    \end{enumerate}
  \item 修改myusers组属性,将xuser1用户加入myusers组。
    \begin{enumerate}
      \item 从“组群”选项卡中选择myusers组,单击工具栏中的“属性”按钮,打开“组群属性”窗口。
      \item 选择“组群用户”选项卡,选中xuser1复选框,设置xuser1用户为myusers组的成员。单击“确定”按钮,返回“用户管理者”窗口。
    \end{enumerate}
  \item 删除temp组。
    \begin{enumerate}
      \item 从“组群”选项卡中选择temp组,单击工具栏中的“删除”按钮,出现确认删除对话框,单击“是”按钮即可。
    \end{enumerate}
\end{enumerate}

\vspace{0.1in}
(九)体验其他桌面环境。

在VirtualBox虚拟机中已经安装好了其他多个Linux发行版(Ubuntu,Kubuntu,Xubuntu,Lubuntu等),涉及多个桌面环境(Unity,KDE,Xfce,LXDE等),依次启动这些操作系统,体验不同的桌面环境。
