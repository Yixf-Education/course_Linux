\chapter{在虚拟机中安装并体验Linux}

%\begin{adjustwidth}{1cm}{1cm}
%\shadowbox{
%\begin{minipage}[l]{0.8\textwidth}
{\itshape
安装Linux对于从未接触过Linux的人而言存在一定的难度。如果在已安装有其他操作系统的计算机上安装Linux,错误的设置有可能导致原有数据全部丢失,造成不可估量的损失。为熟悉Linux的安装过程,可先使用虚拟机来模拟整个安装过程。对于只想尝试Linux的用户而言,在虚拟机中安装Linux也是一个不错的选择。本实验以Ubuntu 14.10和CentOS 7为例学习Linux的安装。

虚拟机(Virtual Machine)不是一台真正的计算机,而是利用真正计算机的部分硬件资源,通过虚拟机软件模拟出一台计算机。虽然只是虚拟机,却拥有自己的CPU等外围设备。现在的虚拟机软件已经能让虚拟机的功能和真正的计算机没有什么区别。用户可以对虚拟机进行磁盘分区、格式化、安装操作系统等操作,而对本身的计算机没有任何影响。

目前,比较常用的虚拟机软件有VMware公司出品的相关产品、甲骨文公司出品的Oracle VirtualBox以及微软公司出品的Virtual PC等。本实验以VirtualBox为例说明虚拟机的使用。VirtualBox是以GNU通用公共许可证(GPL)发布的自由软件,并提供有二进制版本及开放源代码版本的代码,可从其官方网站\href{https://www.virtualbox.org/}{https://www.virtualbox.org/}下载,可运行于Windows和Linux等操作系统环境中。
}
%\end{minipage}
%}
%\end{adjustwidth}

\vspace{0.2in}
\noindent
一、实验要求
\begin{enumerate}
  \item 掌握在VirtualBox中安装Linux(Ubuntu 14.10和CentOS 7)的步骤。
  \item 启动Linux(Ubuntu 14.10和CentOS 7)并进行初始化设置。
  \item 登录不同Linux发行版的桌面环境,了解X Window图形化用户界面。
  \item 掌握注销与关机的方法。
\end{enumerate}

\vspace{0.2in}
\noindent
二、实验准备
\begin{enumerate}
  \item 一台已安装有Windows操作系统和VirtualBox软件的计算机。
  \item 一张Ubuntu 14.10的安装光盘或ISO映像文件。
  \item 一张CentOS 7的安装光盘或ISO映像文件。
\end{enumerate}

\vspace{0.2in}
\noindent
三、实验内容

\vspace{0.1in}
(一)Ubuntu 14.10的安装与启动
\begin{enumerate}
  \item 新建虚拟电脑。
    \begin{enumerate}
      \item 启动VirtualBox软件,进入主界面。单击“新建”图标新建一个虚拟机。
      \item 在“新建虚拟电脑”的“虚拟电脑名称和系统类型”设置中,“名称”中输入虚拟机的名称,如“Ubuntu14.10”,“操作系统”选择“Linux”,“版本”选择“Ubuntu”。
      \item 在“内存大小”的设置中,调整虚拟电脑的内存大小,不少于建议的内存大小。一般使用建议的内存大小即可。
      \item 在“虚拟硬盘”的设置中,选中“现在创建虚拟硬盘”。
      \item 在“虚拟硬盘文件类型”的设置中,使用默认的“VDI”即可。
      \item 在“存储在物理硬盘上”的设置中,选中“固定大小”单选按钮。
      \item 在“文件位置和大小”的设置中,根据需要修改虚拟电脑的存储位置和虚拟硬盘的大小,虚拟硬盘要不小于建议的硬盘大小。单击“创建”按钮,稍等片刻即可完成虚拟电脑的创建。
    \end{enumerate}
  \item 在虚拟机中安装Ubuntu 14.10。
\\ 在VirtualBox虚拟机上安装操作系统时,既可以使用光盘进行安装,也可以利用ISO映像文件进行安装。此处通过ISO映像文件进行安装。
    \begin{enumerate}
      \item 回到VirtualBox的主界面后,选中刚才创建的虚拟电脑。
      \item 点击主界面中的“设置”图标,可单击左侧的“存储”进行设置,为IDE控制器分配光驱,选择Ubuntu 14.10的虚拟光盘,即ISO映像文件。
      \item 回到VirtualBox主界面后,点击“启动”图标启动虚拟电脑。
      \item 虚拟机自动从安装源引导后进入Ubuntu 14.10的启动界面。在“Welcome”界面中,选择安装语言“中文(简体)”,界面随之刷新,选择“安装Ubuntu”进行安装。
      \item 在“准备安装Ubuntu”界面中进行相应的设置。一般默认即可。
      \item 在“安装类型”界面中,选择磁盘分区类型。对于初学者来说,使用默认的“清除整个磁盘并安装Ubuntu”即可。
      \item 在“您在什么地方?”界面中,点选“Shanghai”设置时区。
      \item 在“键盘布局”界面中,一般默认即可。
      \item 在“您是谁?”界面中,设置个人信息。依次填写个人姓名、计算机名、用户名和密码。
      \item 点击“继续”安装系统。系统安装完成后,单击“现在重启”按钮,启动Ubuntu 14.10。
    \end{enumerate}
  \item 启动Ubuntu 14.10。
    \begin{enumerate}
      \item 启动Ubuntu 14.10后,将出现用户登录列表。选择相应的用户,输入对应的密码,回车确认输入。
      \item 用户名和密码验证通过后,进入Unity桌面环境。
    \end{enumerate}
  \item 注销用户。
    \begin{enumerate}
      \item 单击右上角类似齿轮的图标,从中选择“注销…”。
      \item 在弹出的对话框中单击“注销”按钮,将退出Unity桌面环境,屏幕再次显示登录界面,等待新用户登录系统。
    \end{enumerate}
  \item 关机。
    \begin{enumerate}
      \item 单击登录界面右上角类似齿轮的图标,选择“关机…”。
      \item 在弹出的对话框中单击“关机”按钮,系统将依次停止系统的相关服务,直至完全关闭计算机。
    \end{enumerate}
\end{enumerate}

\vspace{0.1in}
(二)CentOS 7的安装与启动

参考Ubuntu 14.10的安装步骤在VirtualBox中安装CentOS 7。

\vspace{0.1in}
(三)体验其他Linux发行版。

在VirtualBox虚拟机中已经安装好了其他多个Linux发行版(Linux Mint,ElementaryOS,Fedora,openSUSE,Debian,Deepin等),依次启动这些操作系统,体验不同的桌面环境。
