\chapter{Perl脚本编程}

%\vspace{0.2in}
\noindent
一、实验要求
\begin{enumerate}
  \item 掌握Perl脚本的语法结构与运行方法。
  \item 掌握Perl读取写入文件的方法。
  \item 了解Perl中获取标准输入的方法。
  \item 熟练使用Perl的基本函数。
  \item 掌握Perl中的判断语句和循环语句。
\end{enumerate}

\vspace{0.2in}
\noindent
二、实验准备
\begin{enumerate}
  \item 安装有CentOS 7的计算机。
\end{enumerate}

\vspace{0.2in}
\noindent
三、实验内容

\vspace{0.1in}
(一)用Perl编写的``Hello World''
\begin{enumerate}
  \item 用文本编辑器创建一个名为hello.pl的文件,内容如下:
\begin{lstlisting}[language=perl]
#!/usr/bin/perl -w
# Classic "Hello World" as done with Perl

print "Hello World!\n";
\end{lstlisting}
  \item 保存文件、退出编辑器之后,修改文件的权限并运行。
\end{enumerate}

\vspace{0.1in}
(二)在Perl中打开文件和目录
\begin{enumerate}
  \item 在主目录中用文本编辑器创建一个名为file.pl的文件,内容如下:
\begin{lstlisting}[language=perl]
#!/usr/bin/perl -w
# Create a file that will have a directory listing of user's home dir

print "About to read user's home directory\n";
# open the home directory and read it
opendir(HOMEDIR, ".");
@ls = readdir HOMEDIR;
closedir(HOMEDIR);

print "About to creat file dirlist.txt with a directory listing of user's home dir\n";
# open a file and write the directory listing to he file
open(FILE, ">dirlist.txt");
foreach $item (@ls) {
  print FILE $item . "\n";
}
close(FILE);
print "All done\n\n";
\end{lstlisting}
  \item 保存文本、退出编辑器之后,修改文件的权限并运行。
  \item 一旦脚本打印出“All done”,请找到dirlist.txt文件并用cat命令显示它的内容。
\end{enumerate}

\vspace{0.1in}
(三)检查来自标准输入的输入
\begin{enumerate}
  \item 在文本编辑器中,输入以下脚本,把它保存为check.pl,并修改权限使得该脚本可执行。
\begin{lstlisting}[language=perl]
#!/usr/bin/perl -T
# check.pl
# A Perl script that checks the input to determine if it contains numeric information

# First set up your Perl environment and import some useful Perl packages
use warings;
use diagnostics;
use strict;
my $result;

# Next check for any input and then call the proper subroutine based on the test
if (@ARGV) {
  # If the test condition for data from standard in is true run the test subroutine on the input data
  $result = &test;
} else {
  # else the test condition is false and you should run an error routine.
  &error;
}

# Now print the results
print $ARGV[0] . " has " . $result . " elements\n\n";
exit(0);

sub test{
  # Get the first array element form the global array ARGV
  # assign it to a local scalar and then test the local variable with a regular expression
  my $cmd_arg = $ARGV[0];
  if ($cmd_arg =~ m/[0-9]) {
    return ('numeric');
  } else {
    # There was a error in the input; generate an error message
    warn "Unknown input";
  }
}

sub error {
  # There was no input; generate an error message and quit
  die "Usage: check.pl, (<INPUT TO CHECK>)";
}
\end{lstlisting}
  \item 在命令行使用不同的参数或者不带任何参数运行该脚本。注意脚本输出的不同文本行取决于提供了什么样的参数:
\begin{lstlisting}[language=bash]
./check.pl
./check.pl 42
./check.pl h2g242
./check.pl hg
\end{lstlisting}
\end{enumerate}

\vspace{0.1in}
(四)Perl的函数与脚本练习

练习理论课中演示的Perl脚本,并尝试对脚本进行调试。
